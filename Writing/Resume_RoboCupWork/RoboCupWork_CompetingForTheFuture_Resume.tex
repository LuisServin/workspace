\documentclass[12pt]{article}
\usepackage[utf8]{inputenc}
\usepackage{amsmath}
\usepackage{amsfonts}
\usepackage{amssymb}

% this is to make it look like a word document.
\usepackage[tmargin=1in,bmargin=1in,lmargin=1.25in,rmargin=1.25in]{geometry}

\title{\vspace{-2ex}RoboCup@Work:\\Compitiendo por la fábrica del futuro\vspace{-2ex}}
\date{\today}
\author{Luis Servín\\ Taller TRA, Universidad Nacional Autónoma de México}

\begin{document}

\maketitle

\section*{Resumen}

RoboCup@Work se enfoca en el uso de manipuladores manipuladores móviles y su integración con equipo de automatización para el desarrollo de tareas relevantes dentro de la industria.

\section{introducción}

Recientemente, tanto la robótica como la automatización industrial están cambiando su campo de atención hacia escenarios en los cuales se ve envuelta la integración de distintos factores como: movilidad y manipulación, integración a gran escala de robots tanto de servicio como industriales, espacios cohabitados entre humanos y robots, así como la colaboración entre múltiples robots y humanos. Todas estas ideas se pueden considerar parte de \emph{la fábrica del futuro} ó FoF por sus siglas en ingles.

Un componente esencial de la FoF son los \textbf{manipuladores móviles} los cuales son una combinan las cualidades de un robot móvil con las capacidades de uno o más manipuladores. Este campo ha llevado al desarrollo de distintas plataformas de investigación. Algunos ejemplos son youBot \cite{bischoff2011kuka} y omniRob por parte de KUKA y rob@worlk por  Fraunhofer IPA.

\bibliographystyle{ieeetr}
\bibliography{bibliography}

\end{document}