\documentclass[12pt]{article}
\usepackage[utf8]{inputenc}
\usepackage{amsmath}
\usepackage{amsfonts}
\usepackage{amssymb}

% this is to make it look like a word document.
\usepackage[tmargin=1in,bmargin=1in,lmargin=1.25in,rmargin=1.25in]{geometry}

\title{\vspace{-2ex}RoboCup@Work:\\Compitiendo por la fábrica del futuro\vspace{-2ex}}
\date{\today}
\author{Luis Servín\\ Taller TRA, Universidad Nacional Autónoma de México}

\begin{document}

\maketitle

\section*{Resumen}

RoboCup@Work se enfoca en el uso de manipuladores manipuladores móviles y su integración con equipo de automatización para el desarrollo de tareas relevantes dentro de la industria.

\section{Introducción}

Recientemente, tanto la robótica como la automatización industrial están cambiando su campo de atención hacia escenarios en los cuales se ve envuelta la integración de distintos factores como: movilidad y manipulación, integración a gran escala de robots tanto de servicio como industriales, espacios cohabitados entre humanos y robots, así como la colaboración entre múltiples robots y humanos. Todas estas ideas se pueden considerar parte de \emph{la fábrica del futuro} ó FoF por sus siglas en ingles.

Un componente esencial de la FoF son los \textbf{manipuladores móviles} los cuales son una combinan las cualidades de un robot móvil con las capacidades de uno o más manipuladores. Este campo ha llevado al desarrollo de distintas plataformas de investigación. Algunos ejemplos son youBot \cite{bischoff2011kuka} y omniRob por parte de KUKA y rob@worlk por  Fraunhofer IPA. Aunque a pesar de todos estos recientes esfuerzos las aplicaciones reales siguen siendo aún raras.

En 2012, un grupo de investigadores concluyó que el progreso en las investigación de manipuladores móviles se podría ver muy beneficiada al organizar una competencia que buscara desarrollar el camino hacia la FoF.

La investigación en si misma podría estar diseñada de tal manera que combinara el atractivo hacia el público como lo es la "Robotic Soccer League", la estructura de la competencia RoboCup@Home y el \emph{benchmarking} tecnológico de la competencia "RoboCup Rescue League"

\section{Temas de Investigación y Retos}

RoboCup@Work busca el uso de robots en ambientes de trabajo y aborda retos de investigación aún abiertos dentro del campo de la robótica industrial y de servicio. Algunos ejemplos de escenario que se buscan dentro de la competencia son:

\begin{itemize}
	\item Carga y descarga de contenedores con objetos de igual o distinto tamaño
\end{itemize}

\bibliographystyle{ieeetr}
\bibliography{bibliography}

\end{document}